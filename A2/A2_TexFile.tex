\documentclass{article}
\usepackage{tikz}
\usetikzlibrary{arrows}
\usepackage[english]{babel}
\usepackage[utf8]{inputenc}
\usepackage{fancyhdr}
 
\pagestyle{fancy}
\fancyhf{}
\rhead{Austin Bursey , Aaron Exley\\ Tim McGill, Joseph Myc}
\lhead{CP468 A2\\November 11th 2019}
\rfoot{Page \thepage}
\begin{document}
\section{Binary Constraints}

\section{Implementation of AC3 (with backtracking) for sudoku }
\begin{verbatim}

from queue import Queue
from copy import deepcopy
import sys
import os
#for the graph at the end
import matplotlib.pyplot as plt

class Node:
    def __init__(self,row,col,value=None, domain=range(1, 10)):
        self.row = row
        self.col = col
        if value is not None: 
            self.value = int(value)
            self.domain = [int(value)]
        else : 
            self.value = value
            self.domain = list(domain)
        

    def __str__(self):
        return '({}, {}, {}, {})'.format(self.row, self.col, self.value, self.domain)

    def __repr__(self):
        return '({}, {}, {}, {})'.format(self.row, self.col, self.value, self.domain)

class Arc: 
    def __init__(self,Xi,Xj):
        self.Xi = Xi
        self.Xj = Xj 

    def evaluate(self):
        '''
        Evaluates arc
        ---------------------------
        returns:
            noSolution: whether or not the state of the puzzle is not solvable
            Checks : Xi != Xj and Xi domain not {}
        '''
        notSolvable = False 
        #enforcing arc consistency Xi != Xj 
        if(self.Xi.value is not None and self.Xi.value == self.Xj.value):
            notSolvable = True 
        elif (self.Xj.value != None and self.Xj.value in self.Xi.domain):
            self.Xi.domain.remove((self.Xj.value))
            if (len(self.Xi.domain)< 0 ): 
                notSolvable = True
        return notSolvable
                
            

        
def addNeighbours(queue,node, puzzle):
     '''
    Adds all Arcs where Xk != Xi given Xi. 
    ---------------------------
    Params (optional):
        queue: a Queue of of arcs to evaluate using AC3
        node: the node Xi you would like to add all neighbhor arcs to
        puzzle: a 2d array of Nodes representing the puzzle
    returns:
        puzzle: a 2d array of Nodes representing the puzzle
        noSolution: a boolean of whether or not the given puzzle is solvable
        noneValueFound: a boolean of whether the given puzzle was fully solved by AC3, returns true if AC3 solved the puzzle
    '''
    currentRow = node.row
    currentColumn = node.col
    #grab neighbhors in nodes row
    i = currentRow
    for j in range(9): 
        if not (currentRow == i and currentColumn == j):
            queue.put(Arc(puzzle[i][j], node ))

    #grab neighbhors in nodes column
    j = currentColumn
    for i in range(9): 
        if not (currentRow == i and currentColumn == j):
            queue.put(Arc(puzzle[i][j], node ))

    #grab neighbhors in box
    row = (currentRow // 3) * 3
    col = (currentColumn // 3) * 3

    for i in range(3) :
        for j in range(3) :
            if not (currentRow == row +i and currentColumn == col+ j):
                queue.put(Arc(puzzle[row + i][col + j], node ))

def AC3(puzzle): 
    '''
    Does AC3 algorithm on Sudoku Puzzle
    ---------------------------
    Params (optional):
        puzzle: a 2d array of Nodes representing the puzzle
    returns:
        puzzle: a 2d array of Nodes representing the puzzle
        noSolution: a boolean of whether or not the given puzzle is solvable
        noneValueFound: a boolean of whether the given puzzle was fully solved by AC3, returns true if AC3 solved the puzzle
    '''
    global qlengths
    queue = Queue()
    
    #fill queue with initial constraints
    for row in puzzle:
        for node in row: 
            currentRow = node.row
            currentColumn = node.col
            #grab neighbhors in nodes row
            i = currentRow
            for j in range(9): 
                if not (currentRow == i and currentColumn == j):
                    queue.put(Arc(node,puzzle[i][j] ))

            #grab neighbhors in nodes column
            j = currentColumn
            for i in range(9): 
                if not (currentRow == i and currentColumn == j):
                    queue.put(Arc( node,puzzle[i][j] ))

            #grab neighbhors in box
            num_row = (currentRow // 3) * 3
            col = (currentColumn // 3) * 3

            for i in range(3) :
                for j in range(3) :
                    if not (currentRow == num_row +i and currentColumn == col+ j):
                        queue.put(Arc(node,puzzle[num_row+ i][col+j] ))
                        
    noSolution = False
    qlengths.append(queue.qsize())
    while (queue.qsize() >  0  and not noSolution): 
        #Get first Node
        arc = queue.get_nowait()
        node = arc.Xi
        
        #get needed attributes
        domainCount = len(node.domain)
        noSolution= arc.evaluate()
        newDomainCount = len(node.domain)

        # this line doesnt cause a problem due to the check within "evaluate" because of this line : if(self.Xi.value is not None and self.Xi.value == self.Xj.value):
        if newDomainCount == 1 :
            node.value = node.domain[0]
        #if domain has been changed , add all neighbors
        if newDomainCount < domainCount:
            addNeighbours(queue,node,puzzle)
        qlengths.append(queue.qsize())

    i=0 
    j= 0 
    #checking if puzzle is solved
    noneValueFound= False
    while (i < 9 and not noneValueFound):
        j = 0
        while(j < 9 and not noneValueFound):
            if puzzle[i][j].value is None: 
                noneValueFound = True
            j+=1
        i +=1 
    return puzzle, not noneValueFound, noSolution




def loadPuzzle(file='./puzzles/easy.csv', num=1, header=True, start=0, givenSolutions=False):
    '''
    Loads a puzzle from a file
    ---------------------------
    Params (optional):
        file: the file to load, defaults to puzzle/easy.csv
        num: the number of puzzles to load, defaults to 1
        header: if the file has a header or not, defaults to True
        start: what line to start reading the puzzle from
    returns:
        puzzle: a 2d array of Nodes representing the puzzle
        solution: a 2d array of Nodes representing the solution of the puzzle
        Note: if num > 1 will return an array of puzzles and and array of solutions
    '''
    with open(file, 'r') as f:
        
        if num == -1 or num > 1:
            puzzles = []
            solutions = []

        for i, line in enumerate(f):
            if header and i == 0:
                start += 1
                continue

            if i < start:
                continue
        
            if num != -1 and i > start + num:
                break
            
            if givenSolutions:
                puzzleAndSol = line.split(',')
            else:
                line = line.replace(',', '')
                puzzleAndSol = [line]
            puzzle = []
            for j in range(9):
                row = []
                for k in range(9):
                    row.append(Node(j, k, None if puzzleAndSol[0][j*(9) + k] == '.' else puzzleAndSol[0][j*(9) + k]))
                puzzle.append(row)

            if givenSolutions:
                solution = []
                for j in range(9):
                    row = []
                    for k in range(9):
                        row.append(Node(j, k, None if puzzleAndSol[1][j*(9) + k] == '.' else puzzleAndSol[1][j*(9) + k]))
                    solution.append(row)

            if num == -1 or num > 1:
                puzzles.append(puzzle)
                if givenSolutions:
                    solutions.append(solution)
    if num == -1 or num > 1:
        return puzzles, solutions
    else:
        return puzzle, solution if givenSolutions else None

def backtrackSearch(puzzle):
    '''
    Performs a backtracking search on a csp sudoku
    ---------------------------
    Param:
        puzzle: A 2d array of Node objects
    returns:
        A solved sudoku puzzle
        A boolean of if the puzzle is solved or not
    '''

    # Find the starting node based on the degree heuristic
    # ie selecting the node with largest amount of constraints
    # since that node will have the largest degree as there will
    # be the most unassigned variables around it.
    firstNode = None
    for row in puzzle:
        for node in row:
            if node.value is None and (firstNode is None or len(firstNode.domain) < len(node.domain)):
                firstNode = node

    # Starts the backtracking
    return backtrack(puzzle, firstNode.row, firstNode.col)

def backtrack(puzzle, row, col):
    '''
    Auxiliary Performs a backtracking search on a csp sudoku
    ---------------------------
    Params:
        puzzle: A 2d array of Node objects
        row: The row of the current node
        col: The col of the current node
    returns:
        A solved sudoku puzzle
        A boolean for if the puzzle is solved or not
    '''
    
    # Check if the puzzle is finished, if it is we are done
    # and collapse the call stack
    if complete(puzzle):
        return puzzle, True

    # Get the order of the domain using the
    # least consraining value heuristic
    domainOrder = order(row, col, puzzle)

    for value in domainOrder:

        # Check if the current value in the domain is consistant
        # with the constraints of the sudoku
        # Should always be consisitant
        if valid(puzzle, row, col, value):

            # Store a copy of the current state for
            # the backtracking
            state = deepcopy(puzzle)

            # Update the value of the current node
            puzzle[row][col].value = value

            # Make the updated puzzle arc consistant
            puzzle, completed, noSolution = AC3(puzzle)

            # If the puzzle still has a solution
            # We can continue, Otherwise we move on 
            if not noSolution:
                if not completed:
                    # Figure out which node we should check next using
                    # MRV
                    nextNode = getNextNode(puzzle)

                    # Call the next node
                    puzzle, completed = backtrack(puzzle, nextNode.row, nextNode.col)

                    # We returned from the backtracking
                    # if completed then we are done
                    # collapse the call stack
                    if completed:
                        return puzzle, completed

                else:
                    return puzzle, completed
        # we returned from the backtracking
        # or the value is invalid
        # so restore the starting state
        # remove the value from the domain
        # and continue to the next value in the domain
        puzzle = state
        puzzle[row][col].domain.remove(value)
    puzzle[row][col].value = None

    # This value had no values in its domain that worked
    # Backtrack to previous node
    return puzzle, False

def valid(puzzle, row, col, value):
    '''
    Checks if a value is valid with the contraint
    ---------------------------
    Params:
        puzzle: A 2d array of Node objects
        row: The row of the current node
        col: The col of the current node
        value: The value you are checking that works
    returns:
        True if the value is valid, false otherwise
    '''
    node = puzzle[row][col]

    # Row
    col = node.col
    for row in range(9):
        if puzzle[row][col].value == value:
            return False
    
    # Column
    row = node.row
    for col in range(9):
        if puzzle[row][col].value == value:
            return False
    # Box
    row = (node.row // 3) * 3
    col = (node.col // 3) * 3
    for i in range(3):
        for j in range(3):
            if puzzle[row + i][col + j].value == value:
                return False
    return True

def order(row, col, puzzle):
    '''
    returns the order of the domain to check using the least
    contraining value heuristic
    ---------------------------
    Params:
        row: The row of the current node
        col: The col of the current node
        puzzle: A 2d array of Node objects
    returns:
        The domain as a new array in the order to use
    '''

    node = puzzle[row][col]

    order = []
    for value in node.domain:
        affectedValues = 0

        boxRow = (node.row // 3) * 3
        boxCol = (node.col // 3) * 3
        # Row
        col = node.col
        for row in range(9):
            if boxRow <= row < boxRow + 3:
                continue
            if value in puzzle[row][col].domain:
                affectedValues += 1
        
        # Column
        row = node.row
        for col in range(9):
            if boxCol <= col < boxCol + 3:
                continue
            if value in puzzle[row][col].domain:
                affectedValues += 1
        
        # Box
        for i in range(3):
            for j in range(3):
                if value in puzzle[boxRow + i][boxCol + j].domain:
                    affectedValues += 1

        order.append((value, affectedValues))
        
    order = sorted(order, key=lambda x: x[1])
    return [x[0] for x in order]


def complete(puzzle):
    '''
    Checks if the puzzle is solved
    ---------------------------
    Params:
        puzzle: A 2d array of Node objects
    returns:
        True if every node has a value, false otherwise
    '''
    for i in range(9):
        for j in range(9):
            if puzzle[i][j].value == None:
                return False

    return True


def getNextNode(puzzle):
    '''
    returns the next node to check using MRV
    ---------------------------
    Params:
        puzzle: A 2d array of Node objects
    returns:
        The next node to search
    '''
    nextNode = None
    for row in puzzle:
        for node in row:
            if node.value is None and (nextNode is None or len(nextNode.domain) > len(node.domain)):
                nextNode = node
    return nextNode

def print_board(puzzle, detailed=False):

    print('- ' * 13)
    for row in puzzle:
        print('|', end=' ')
        for col in row:
            if detailed:
                domain = ''.join(str(x) for x in col.domain)
                print("[{} ({:9s})]".format(col.value if col.value else '.', domain), end=' ')
            else:
                print(col.value if col.value else '.', end=' ')
            if col.col % 3 == 2:
                print('|', end=' ')
        print()
        if col.row % 3 == 2:
            print('- ' * 13)

def print_board_and_sol(puzzle, solution):

    print("{:^25s}{:20s}{:^25s}".format("Original Puzzle", "", "Solved After AC3"))
    print('-' * 25, end='')
    print(' ' * 20, end='')
    print('-' * 25)
    for rowp, rows in zip(puzzle, solution):
        print('|', end=' ')
        for col in rowp:
            print(col.value if col.value else '.', end=' ')
            if col.col % 3 == 2:
                print('|', end=' ')

        print(' ' * 19, end='| ')
        for col in rows:
            print(col.value if col.value else '.', end=' ')
            if col.col % 3 == 2:
                print('|', end=' ')
        print()
        if col.row % 3 == 2:
            print('-' * 25, end='')
            print(' ' * 20, end='')
            print('-' * 25)

def print_board_and_sol_and_ac3(puzzle, ac3, solution):

    print("{:^25s}{:20s}{:^25s}{:20s}{:^25s}".format("Original Puzzle", "", "After AC3", "", "Solution"))
    print('-' * 25, end='')
    print(' ' * 20, end='')
    print('-' * 25, end='')
    print(' ' * 20, end='')
    print('-' * 25)
    for rowp, rowac, rows in zip(puzzle, ac3, solution):
        print('|', end=' ')
        for col in rowp:
            print(col.value if col.value else '.', end=' ')
            if col.col % 3 == 2:
                print('|', end=' ')

        print(' ' * 19, end='| ')
        for col in rowac:
            print(col.value if col.value else '.', end=' ')
            if col.col % 3 == 2:
                print('|', end=' ')

        print(' ' * 19, end='| ')
        for col in rows:
            print(col.value if col.value else '.', end=' ')
            if col.col % 3 == 2:
                print('|', end=' ')
        print()
        if col.row % 3 == 2:
            print('-' * 25, end='')
            print(' ' * 20, end='')
            print('-' * 25, end='')
            print(' ' * 20, end='')
            print('-' * 25)

qlengths = []
if __name__ == "__main__":
    q = Queue()

    num = int(sys.argv[1]) if len(sys.argv) > 1 else 1

    puzzles, _ = loadPuzzle(file='./puzzles/random.txt',num=num, header=True)

    if num == 1:
        puzzles = [puzzles]

    for i, p in enumerate(puzzles, start=1):
        qlengths = []
        
        original_puzzle = deepcopy(p)

        finishedPuzzle, completed, noSolution = AC3(p)

        if len(puzzles) > 1:
            print("{}:".format(i))

        if completed:
            print("Sudoku solved using AC3")
            print_board_and_sol(original_puzzle, finishedPuzzle)
        elif noSolution:
            print('No Solution!')
            print_board(original_puzzle)
        else:

            print('Board used Backtracking')
            ac3_puzzle = deepcopy(finishedPuzzle)
            finishedPuzzle2, finished = backtrackSearch(finishedPuzzle)
            print_board_and_sol_and_ac3(original_puzzle, ac3_puzzle, finishedPuzzle2)
            if not finished:
                print("Backtracking Failed to find a solution")
                print_board(finishedPuzzle2)

        #ploting the queue lengths
        if not os.path.exists('queue_lengths'):
            os.makedirs('queue_lengths')
        plt.plot(qlengths)
        plt.xlim(left=0)
        plt.ylim(bottom=0)
        plt.ylabel('Length of the Queue')
        plt.xlabel('Step count')
        plt.savefig('queue_lengths/Sudoku-Queue-length-plot-{}.png'.format(i))
        # plt.show()
        plt.close()
\end{verbatim}
\end{document}